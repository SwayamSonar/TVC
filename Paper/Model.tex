\section{Models} \label{sec:Models}
%
In this section, different models of varying complexities will be proposed as applications, such as \gls{lqr}, require modifications of the model derived in the previous section,~\fullrefnocomma{sec:Dynamics}.
%
\subsection{First Model}
This is the most accurate model of the system which will be used in the simulation to evaluate the performance of a proposed \gls{gnc} scheme.
\begin{align}
    [\dot{\m \omega^b}]^{bn} &= (\m I^b)\inv ( \mathbf{AeroMoments} + \mathbf{Actuator\ Moments}(t) - [\m \omega^b]^{bn} \times \m I^b [\m \omega^b]^{bn} ) \\
    \label{eq:qdot}
    \dot{\q}^{bn} &=\frac{1}{2} \q^{bn}
    \begin{bmatrix}
        0 & [\m \omega^b]^{bn}
    \end{bmatrix}\tran \\
    [\dot{\m v}^n]^{nn} &= \m R^{nb} m\inv \left( \mathbf{AeroForces}^b + \mathbf{Thrust}^b \right) +
    \begin{bmatrix}
        0 & 0 & -g
    \end{bmatrix}\tran \\
    \label{eq:rdot}
    [\dot{\m r}^n]^n &= [\m v^n]^n 
\end{align}
%
\subsection{Second Model} 
\label{Position Model}
This model is for used for translational control using linear techniques such as \gls{lqr}.

As little translational control can be achieved in the z axis and it would be difficult to incorporate this into state space form, so the z axis is not included. 
Therefore, the weight force is omitted.
Additionally the quaternion error term, $\qv$ is used instead of $\q$. 
The choice of $\qv$ is explained in \fullref{subsec:ErrorTerms}.
\begin{align}
    \label{eq:alpha}
    [\dot{\m \omega^b}]^{bn} &= (\m I^b)\inv (\mathbf{Actuator\ Moments}(t) - [\m \omega^b]^{bn} \times \m I^b [\m \omega^b]^{bn} ) \\
    \label{eq:qdotv}
    \dot{\q}^{bn}_v &= \left( \frac{1}{2} \q^{bn}
    \begin{bmatrix}
        0 & [\m \omega^b]^{bn}
    \end{bmatrix}\tran\right)_v\\
    \label{eq:qintv}
    \frac{d}{dt} \int \q^{bn}_v &= \q^{bn}_v \\
    [\dot{\m v}^n_{x,y}]^{nn} &= \left( \m R^{nb} m\inv \mathbf{Thrust}^b \right)_{x,y}\\
    [\dot{\m r}^n_{x,y}]^n &= \m v^n_{x,y}
\end{align}
%
\subsection{Third Model} 
\label{Attitude Model}
This model is for used for \gls{attitude} control using linear techniques such as \gls{lqr} and is based on the previous model, \modelref{Position Model} so \crefrange{eq:alpha}{eq:qintv} are used.
%
\subsection{Fourth Model} 
\label{KF Model}
This model, in contrast to the \modelref{Position Model} includes the translation variables in the z axis and $\q_w$ so that it can be used to build a state estimator or observer.
Let the gyroscope bias be $\m \omega_b$ and the accelerometer bias $\m a_b$.

Using equations~\eqref{eq:qdot},~\eqref{eq:rdot},~\eqref{eq:alpha}
\begin{align}
    \dot{\m \omega^b}_b &= 0\\
    \dot{\m a}^b_b &= 0 \\
    [\dot{\m v}^n]^{nn} &= \m R^{nb} m\inv \mathbf{Thrust}^b + 
    \begin{bmatrix}
        0 & 0 & -g
    \end{bmatrix}\tran  \\
\end{align}
%
\subsection{Fifth Model} 
\label{Accurate Actuator}
Different actuator models are needed, e.g. for controllers which take into account the time delay in the actuator and those that don't. 

This model is the most accurate model of the system and will be used in the simulation to evaluate the performance of a proposed \gls{gnc} scheme.

\Crefrange{eq:Actuator1}{eq:Actuator4} describe this model.
%
\subsection{Sixth Model} 
\label{No Saturation Actuator}
This actuator model ignores the saturation and can be continuous or discrete with any sample time by appropriately modifying \eqref{eq:Actuator1} and \eqref{eq:Actuator4}.

This model is appropriate for controllers that cannot be tuned/ developed with models including saturation terms such as a \gls{lqr} or \gls{pid} controller.

Using equations~\eqref{eq:Actuator1},~\eqref{eq:Actuator3},~\eqref{eq:Actuator4} 
\begin{align}
    \label{eq:noSaturationActuator}
    \aforce &= \fm f(t)
\end{align}
%
\subsection{Seventh Model} 
\label{No Time Delay Actuator}
This actuator model additionally assumes that there is no time delay in the actuators compared to the~\modelrefnocomma{No Saturation Actuator}.

Using equations~\eqref{eq:noSaturationActuator} and~\eqref{eq:Actuator4}.
\begin{align}  
    \label{eq:ActuatorSimple1}
    \fm f(t) &= \cotto\inv\begin{bmatrix}
        - u_y(t) \\
         u_x(t)
    \end{bmatrix} \\
    \amom &= \m u(t)
\end{align}