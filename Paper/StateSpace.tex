\subsection{State Space}
\label{sec:StateSpace}

A state space representation is when a system is described using first order differential or difference equations. To express a continuous state space representation in matrix vector form the equations must be linear:
\begin{align}
    \dot{\m x}(t) &= \m A(t)\m x(t) + \m B(t) \m u(t) \\
    \m y(t) &= \m C(t)\m x(t) + \m D(t) \m u(t)
\end{align}
Most controllers require \gls{lti} systems which are described as such in a state space representation if described using differential equations:
\begin{align}
    \label{eq:ltistate}
    \dot{\m x}(t) &= \m A \m x(t) + \m B \m u(t) \\
    \label{eq:ltioutput}
    \m y(t) &= \m C \m x(t) + \m D \m u(t)
\end{align}
If the state space system is implemented on a digital system, if the actuators are discrete or difference equations are used the state space system may be described in a discrete representation.
The \gls{lti} discrete state space representation is expressed as such:
\begin{align}
    \m x_{k+1} &= \m A \m x_k + \m B \m u_k \\
    \m y_k &= \m C \m x_k + \m D \m u_k
\end{align}

To convert from a time varying state space representation to a time invariant representation, the representation at an operating point can be taken to hold for the entire range of values.
Typically, the equilibrium points are used as the system is likely to stabilize around that point~\cite{MITFEEDBACK}. This means the state will be close to the equilibrium point most of the time so the representation stays reasonably accurate. 
This has disadvantages, e.g. if the state deviates from the operating points for extended periods or for a short time if the plant is strongly non-linear the \gls{lti} representation will be inaccurate.

Many systems are non-linear, so to apply linear techniques such as \gls{lqr} non-linear models need to be linearised about operating points $\bar{\m x}, \bar{\m u}, \bar{\m y}$. 
We often wish to use linear techniques as they are simpler, linear control theory is more developed than non-linear control theory and control designers will have more knowledge about linear techniques.
Additionally, non-linear control techniques are typically only useful for certain non-linear systems whilst linear techniques can work for a wide range of systems.
A non-linear system expressed as:
\begin{align}
    \dot{\m x}(t) &= f(\m x(t), \m u(t), \bullet) \\
    \m y(t) &= g(\m x(t), \bullet)
\end{align}
Can be linearised as such~\cite{Bla2010}:
\begin{align}
    \tilde{\m x} &= \m x - \bar{\m x} \\
    \tilde{\m u} &= \m u - \bar{\m u} \\
    \tilde{\m y} &= \m y - \bar{\m y}
\end{align}
The taylor series expansion of $\dot{x}$ and $y$ are as follows:
\begin{align}
    \dot{\m x} &= f(\bar{\m x}, \bar{\m u}, \bar{\bullet}) + \frac{\partial f(\bar{\m x}, \bar{\m u}, \bar{\bullet})}{\partial \m x} \tilde{\m x} + \frac{\partial f(\bar{\m x}, \bar{\m u}, \bar{\bullet})}{\partial \m u} \tilde{\m u} \dots \\
    \m y &= g(\bar{\m y}, \bar{\bullet}) + \frac{\partial g(\bar{\m y}, \bar{\m u}, \bar{\bullet})}{\partial \m y} \tilde{\m y} + \frac{\partial g(\bar{\m y}, \bar{\bullet})}{\partial \m u} \tilde{\m u} \dots
\end{align}
To obtain a linear state system use only the first order terms.
\begin{align}
    \dot{\tilde{\m x}} &= \frac{\partial f(\bar{\m x}, \bar{\m u}, \bar{\bullet})}{\partial \m x} \tilde{\m x} + \frac{\partial f(\bar{\m x}, \bar{\m u}, \bar{\bullet})}{\partial \m u} \tilde{\m u} \\
    \tilde{\m y} &= \frac{\partial g(\bar{\m y}, \bar{\bullet})}{\partial \m y} \tilde{\m y} + \frac{\partial g(\bar{\m y}, \bar{\bullet})}{\partial \m u} \tilde{\m u}
\end{align}

In order to obtain better performance, gain scheduling can be used where multiple operating points are chosen to generate linear controllers.
Depending on the current state, controllers can be appropriately selected or interpolated between.

Time delays present in a system may be approximated using a padé approximation which may add additional states and so a state observer or estimator is also required.

A system is observable if the current state can be estimated using the output for all possible states and control vectors.
A state space system of the form \eqref{eq:ltistate}, \eqref{eq:ltioutput} is observable if the rank of $\m{\mathcal{O}}$ is the same as the number of state variables \cite{fri2005}.
\begin{align}
    \m{\mathcal{O}} = \begin{bmatrix}
        \m C \\
        \m C \m A \\
        \m C \m A^2 \\
        \vdots \\
        \m C \m A^{n-1}
    \end{bmatrix}
\end{align}

A system is controllable if any state can be achieved from any initial state in a finite time.
A state space system of the form \eqref{eq:ltistate}, \eqref{eq:ltioutput} is controllable if the rank of $\m \zeta$ is the same as the number of state variables \cite{fri2005}.
\begin{align}
    \m \zeta = \begin{bmatrix}
        \m B \\
        \m A \m B \\
        \m A^2 \m B \\
        \vdots \\
        \m A^{n-1} \m B
    \end{bmatrix}
\end{align}

A system can be controlled using \gls{lqr} or pole placement if the \gls{lti} state space representation is controllable.
Pole placement and \gls{lqr} are similar as they both set the eigenvalues of the system, which determines the system's dynamics and so stability. 
However, pole placement is given eigenvalues whilst \gls{lqr} determines the optimal eigenvalues and gain matrix, $\m K$, minimising the quadratic cost function, $J$, where $\m u = -\m K \m x$ and $\dot{ \m x } = (\m A - \m B \m K) \m x$~\cite{fri2005}.

The infinite-horizon, continuous-time \gls{lqr} defines the cost, $J$, as \begin{align}
    J = \int_{0}^{\infty} (\m x\tran \m Q \m x + \m u\tran \m R \m u + 2 \m x\tran \m N \m u) \,\mathrm{d}t.
\end{align}
The infinite-horizon, discrete-time \gls{lqr} defines the cost, $J$, as 
\begin{align}
    J = \sum_{k = 0}^{\infty} (\m x_k\tran \m Q \m x_k + \m u_k\tran \m R \m u_k + 2 \m x_k\tran \m N_k \m u).
\end{align}
$\m K$ can be found using the algebraic Riccati equation.

Additionally, more advanced controllers can be used that are designed for non-linear, time varying systems or that can account for constraints, such as iLQR or \gls{mpc}, can be designed using a state space model.
However, these are more computationally expensive than pole placement and \gls{lqr} and may be less robust.